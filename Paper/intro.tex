\section{Introduction}
\label{sec:intro}

%1. Why UQ is important for this problem?
%2. Challenges pertaining to UQ. Include a table for exponential scaling. 
%3. What has been done in previous UQ studies for atomistic level simulations.
%4. How we tried to address the challenge in our previous work.
%5. How is this work different?
%6. Key contributions.
%7. Organization of the paper.

Non-equilibrium molecular dynamics (NEMD) simulations are commonly used to investigate
bulk thermal conductivity of non-metallic elements such as carbon, silicon, and
germanium~\cite{Dumitrica:2010}. The system is subjected to either a heat flux or a temperature
gradient by means of thermostatting. The resulting steady-state temperature gradient in the
former and heat exchange between the thermostats in the latter is recorded. The thermal
conductivity at a given system size is then estimated using Fourier's law. In the
so-called direct method~\cite{Schelling:2002,Turney:2009,Zhou:2009,Landry:2009,
McGaughey:2006,Ni:2009,Shi:2009,Wang:2009,Papanikolaou:2008},
the thermal conductivity is estimated at multiple values of the
system size. The inverse of thermal conductivity ($\kappa^{-1}$) is plotted against the inverse of 
system size ($L^{-1}$) and a linear extrapolation procedure is used to estimate the y-intercept
of the plot. The inverse of the y-intercept is regarded as the bulk thermal conductivity
of the system since it corresponds to an infinitely large system (in theory). 

Although widely used, severe limitations are associated with the direct method.
The validity of the linear extrapolation procedure is not well established. Recent
investigations have revealed the existence of a non-linear trend in the $\kappa^{-1}$-$L^{-1}$
relationship especially at large values of $L$~\cite{Sellan:2010} for Si. Additionally, 
thermal conductivity estimate for a given size depends upon the choice of a potential
function and associated values of its parameters. Specifically for Si, the Stillinger-Weber (SW)
inter-atomic potential ($\Phi$) is commonly used in a large number of studies, as
%
\be
\Phi = \sum\limits_{i,j(i<j)}\phi_2(A,B,p,q,\alpha)\hspace{1mm}+\sum\limits_{i,j,k(i<j<k)}\phi_3(\lambda,\gamma)
\ee
%
However, according
to the methodology presented by Stillinger and Weber in~\cite{Stillinger:1985},
the following shortcomings must be noted:
%
\begin{itemize}
\item The SW potential function accounts for the second-order ($\phi_2$) and
third-order ($\phi_3$) atomic 
interactions. However, this representation can be inadequate in situations where 
higher order interactions become significant.  
\item Nominal values of the SW potential parameters were estimated using a 
limited search in a 7-dimensional parameter space. Regression-based parameter
estimates relied on the available set of experimental data while ensuring structural
stability. Hence, the estimates are tightly coupled with the set of data used for
calibration, and did not account for the presence of measurement uncertainty. 
\item Noise inherent in MD predictions can also be significant and
was not accounted for in the analysis. 
\end{itemize}
%
Hence, using the same set of nominal values for a wide range of Si-based systems and
applications is not ideal. It is therefore important to attribute uncertainty to the values and
investigate its impact on NEMD predictions. 

This paper presents a computational framework aimed at dimension reduction
for enabling the propagation of uncertainty from the SW potential parameters to the
quantity of interest (QoI) i.e. the bulk thermal conductivity of Si,
in an efficient manner. For this purpose, we perform NEMD simulations of a Si bar
as discussed later in~\ref{sub:nemd}. The
motivation for input-space dimension reduction stems from exponential scaling in
computational effort with the number of parameters, illustrated in Table~\ref{tab:effort}. 
%
\newcommand{\ra}[1]{\renewcommand{\arraystretch}{#1}}
\begin{table}[htbp]
\centering
\ra{1.3}
\begin{tabular}{@{}ccc@{}}\toprule
Parameters & Model Runs & Time (1h/run)\\
\bottomrule
1 & 10 & 10 hours \\
2 & 10$^2$ & 4.2 days \\
3 & 10$^3$ & 1.4 months \\
4 & 10$^4$ & 1.2 years \\
5 & 10$^5$ & 11.6 years \\
6 & 10$^6$ & 115.7 years \\
7 & 10$^7$ & 1157.41 years \\
\bottomrule
\end{tabular}
\caption{Exponential scaling of computational effort with the number of uncertain parameters.}
\label{tab:effort}
\end{table}
%
Here, we consider that in order to perform uncertainty quantification (UQ),
an average of 10 runs along each input dimension
is needed, and each run takes approximately an hour. Therefore, the time required to 
perform the analysis in a 7-dimensional input space is $\mathcal{O}(10^3)$ years which makes it
intractable. Note that the considered compute times do not account for simulation queue times that further 
increase the time required for a run. 
Although computational gains can be realized using parallelism and sparse grids
for efficient surrogate construction~\cite{Ma:2009,Constantine:2012,Petvipusit:2014,Vohra:2014}, 
the issue of exponential scaling would persist.  
Hence, we aim to tackle the computational expense by focusing our efforts on reducing the 
dimensionality of the problem.
In our earlier effort, we presented a strategy~\cite{Vohra:2018b}
for constructing a surrogate for thermal conductivity dependence on the SW parameters
in a reduced space, evaluated using the derivative-based global sensitivity measures
(DGSMs)~\cite{Vohra:2018a}. The focus there was on determining the relative importance of the
parameters and fixing the unimportant parameters at their nominal values. On the other hand, the
framework presented in this study aims to identify key directions in the input space along which the
QoI predominantly varies. The set of directions constitute the so-called 
\textit{active subspace}~\cite{Constantine:2015}.  

The active subspace methodology relies on estimating the partial derivative of the QoI 
with respect to each uncertain input. This requirement poses a couple of limitations on its
applicability: (1) The QoI is required to be differentiable in the considered input domain, and (2)
estimating the derivative typically requires additional model evaluations which
imposes a computational burden. Fortunately, in this case,
the thermal conductivity is observed to exhibit a smooth dependence on the SW parameters. 
To mitigate the challenge pertaining to derivative estimation, we exploit 
a regression-based approach for evaluating the active subspace~\cite{Vohra:2018c}. 
In this approach, the gradient is estimated
using a linear regression fit to the available set of model evaluations and therefore does not 
require additional evaluations at neighboring points as in the case of perturbation-based
methods such as finite difference. Hence, the regression-based approach seems like a suitable 
choice for the present application involving molecular dynamics simulations since the
perturbation-based approaches although more accurate would be intractable owing to the
associated computational effort. Moreover, it was shown for a chemical kinetics application
in~\cite{Vohra:2018c} that the regression-based approach led to reasonable estimates of
 the statistics of the QoI due to input uncertainties.
As mentioned earlier, a computationally tractable approach for propagating the uncertainty
from the potential parameters to the bulk thermal conductivity of Si is the prime focus of this  
effort. In fact, we have
shown that using a regression-based approximation to the gradient, a 1-dimensional
active subspace is obtained which is also verified for accuracy. Additionally, the global 
sensitivity measures, obtained using the 1-dimensional active subspace, are found to be consistent with 
our earlier estimates based on DGSMs~\cite{Vohra:2018b}. 

The remainder of this article is organized as follows. In section~\ref{sec:bg}, we provide details
pertaining to the NEMD simulation in \ref{sub:nemd}, and a brief background on active subspaces
in~\ref{sub:as}. 
Details pertaining to the computational framework are
provided in section~\ref{sec:method}. Results based on implementing the framework to
compute the active subspace
are presented in section~\ref{sec:results}. Finally, useful conclusions are drawn based on this
work in section~\ref{sec:conc}.
































