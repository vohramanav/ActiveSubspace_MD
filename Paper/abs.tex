\section*{Abstract}
This paper develops an efficient methodology for both forward and inverse 
problems in uncertainty quantification with respect to molecular dynamics simulation.
Specifically, our objectives are to investigate the impact of uncertainty in the
Stillinger-Weber (SW) potential parameters on NEMD-based predictions of bulk
thermal conductivity of silicon (forward problem), and perform a Bayesian calibration
of these parameters using experimental data (inverse problem). 
However, both analyses typically require tens of thousands of model evaluations
and therefore, relying purely on atomistic simulations would be impractical.
The common strategy of building a surrogate model in the space of the uncertain
parameters is also unaffordable due to the need for many training evaluations
using atomistic simulations. Therefore, computational effort 
is minimized in this paper by reducing the
dimensionality of the input space of the surrogate model by first computing
the so-called active subspace.
Active subspace computation, however, requires gradient information of
the model output with respect to the uncertain inputs which can be
computationally challenging. To overcome this challenge, we estimate the
gradient using a linear regression fit to the available set of
model evaluations as opposed to perturbation-based gradient estimation
(e.g. finite difference, adjoint-based methods etc.) 
that would require additional model runs. Moreover, 
we employ an iterative strategy to avoid unnecessary computations once
a converged active subspace is obtained.
The active subspace thus computed in an efficient manner is found to be
1-dimensional, indicating enormous scope for dimension reduction and
computational savings. A surrogate model is 
then built in the 1-dimensional subspace to help quantify
the variability of the bulk thermal conductivity, and is shown to have
reasonable accuracy. The
active subspace is also used to perform efficient global sensitivity analysis (GSA)
of the SW parameters. Finally, we demonstrate the use of active subspace-based
surrogate model for fast calibration of SW parameters in a Bayesian setting.  
