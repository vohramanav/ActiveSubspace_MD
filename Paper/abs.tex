\section*{Abstract}
Bulk phonon transport in silicon (Si) is commonly investigated using the
Stillinger-Weber (SW) interatomic potential in non-equilibrium molecular
dynamics (NEMD). Nominal estimates
of the SW parameters, however, are based on a limited search
using a sparse set of measurements and constrained by structural stability
of the considered system in~\cite{Stillinger:1985}. It is therefore
critical to study the impact of uncertainty associated with nominal
parameter values on bulk phonon transport using NEMD. 
However, the analysis typically requires tens of thousands of model evaluations
and hence, relying purely on atomistic simulations would be impractical.
In this work, we focus on propagating the uncertainty from SW parameters
to bulk thermal conductivity estimates using NEMD simulations of a Si bar.
Computational effort is minimized by reducing the
dimensionality of the input space by computing the so-called active subspace.
 Active subspace computation requires gradient information of
the model output with respect to the uncertain inputs which imposes a 
computational burden. To overcome this, we employ an iterative strategy
that estimates the model gradient using a regression-based local linear
approximation and hence, does not require model evaluations. 
The active subspace thus computed in an efficient manner is found to be
1-dimensional, indicating enormous scope for dimension as well as
associated computational effort reduction. A surrogate model is used to
represent the variability of the bulk thermal conductivity in the subspace.
A probability density function (PDF) based on surrogate predictions at 10$^5$
random samples in the
7-dimensional input space is shown to capture the mean, mode, and the
uncertainty in predictions with reasonable accuracy. Finally, the
active subspace is used to perform a global sensitivity analysis (GSA)
of the SW parameters.  
