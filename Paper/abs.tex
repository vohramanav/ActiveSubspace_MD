\section*{Abstract}
We are interested in investigating the impact of uncertainty in the
Stillinger-Weber potential parameters on NEMD-based predictions of bulk
thermal conductivity of silicon. 
However, the analysis typically requires tens of thousands of model evaluations
and therefore, relying purely on atomistic simulations would be impractical.
Hence, we focus on an efficient approach for propagating the 
uncertainty from SW parameters
to bulk thermal conductivity estimates using NEMD simulations of a Si bar.
Computational effort is minimized by reducing the
dimensionality of the input space by computing the so-called active subspace.
Active subspace computation, however, requires gradient information of
the model output with respect to the uncertain inputs which can be
computationally challenging. To overcome this challenge, we estimate the
gradient using a linear regression fit to the available set of
model evaluations as opposed to a perturbation-based gradient estimation
(e.g. finite difference, adjoint-based methods etc.) 
that would require additional model runs. Moreover, 
we employ an iterative strategy to avoid unnecessary computations once
a converged active subspace is obtained.
The active subspace thus computed in an efficient manner is found to be
1-dimensional, indicating enormous scope for dimension as well as
associated computational effort reduction. A surrogate model is 
then built in the 1-dimensional subspace to help quantify
the variability of the bulk thermal conductivity, and is shown to have
reasonable accuracy. The
active subspace is also used to perform efficient global sensitivity analysis (GSA)
of the SW parameters. Moreover, we demonstrate its applicability for accelerating
parameter estimation in a Bayesian setting.  
