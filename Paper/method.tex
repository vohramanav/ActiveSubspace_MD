\section{Methodology}
\label{sec:method}

As mentioned earlier in section~\ref{sec:intro}, the gradient-free approach is used in this work
to compute the active subspace for enabling efficient propagation of the uncertainty from SW
parameters to thermal conductivity estimates based on NEMD simulations. The gradient-free
approach yields a computational advantage by not relying on model evaluations for estimating
the gradients required for estimating $\hat{\mat{C}}$ in~\eqref{eq:chat}. 
Instead, it involves a regression-based local linear approximation of the model output as
discussed in~\cite{Constantine:2015} (Algorithm 1.2). In this work, however, we implement the
gradient-free approach in an iterative manner to avoid extraneous model evaluations once
convergence of $\hat{\mat{C}}$ has been established as discussed further below. 

