\section{Conclusion}
\label{sec:conc}

Bulk thermal conductivity predictions (QoI) for Si using NEMD are strongly dependent upon the
choice of nominal values of the SW parameters. However, the calibration process for the
parameters as discussed by Stillinger and Weber in~\cite{Stillinger:1985} was not
necessarily robust as it was based on a limited search focused on agreement with the 
available set of measured data, and ensuring the structural stability of the considered system. 
It did not account for the presence of measurement error, noise inherent in MD simulations,
and prior uncertainty associated with the parameters. It is therefore critical to
investigate the impact of uncertainty in the calibrated SW parameters on model predictions.
However, as highlighted in this work, it would be impractical to completely rely on
atomistic simulations for this purpose. Hence, in this study, we focused our efforts on
discovering the so-called active subspace that predominantly captures the variability
in bulk thermal conductivity predictions as a function of fewer independent variables
compared to the original input space. Efficient computation of the active subspace
was accomplished by estimating the gradient of the bulk thermal conductivity with
respect to uncertain SW potential parameters, by means of regression. Moreover, the
computations were performed in an iterative manner whereby, new evaluations at a 
small set of samples were generated to refine the active subspace while assessing 
its convergence at each step.
Based on the implementation of this approach and the results
presented in this work, the following conclusions can be drawn:
%
\begin{enumerate}
\item The active subspace computation using the iterative gradient-free approach
required only 25 NEMD simulations in the 7-dimensional input space using a tolerance, 
$\tau$~=~0.1.

\item A 1-dimensional active subspace was obtained indicating enormous scope for
computational gains. In fact, conventional
approaches for the present application quickly become prohibitive for
the purpose of UQ. 

\item The 1-dimensional surrogate in the active subspace was shown to be reasonably accurate
by estimating the L$^2$ norm of the error between its predictions and the available set of
model evaluations. Additionally, the PDF based on the 1-dimensional surrogate was 
observed to capture the mean, mode, and the uncertainty in the QoI with reasonable accuracy.

\item Global sensitivity analysis results for Sobol' indices using the surrogate, and the
normalized activity scores were consistent with each other. Specifically, the SW parameters:
$\gamma$, $\alpha$, and $\lambda$ were found to be major contributors to the
uncertainty in the bulk thermal conductivity of Si.

\item It was shown that the active subspace could help accelerate the process
of Bayesian calibration through dimension reduction and the use of surrogate 
(as opposed to NEMD simulations) in order to map physical parameters to the QoI.    

\end{enumerate}
%
The large uncertainty in the bulk thermal conductivity based on NEMD predictions due to
a 10$\%$ perturbation in the nominal SW parameter values in Figure~\ref{fig:level2} underscores
the importance of UQ in atomistic simulation predictions. Successful implementation of the
computational framework aimed at dimension reduction presented in this work is remarkably
encouraging. Future extensions to this work would involve enhancing the efficiency as well
as the accuracy of the computational framework through its implementation to more complex,
higher dimensional applications. 
