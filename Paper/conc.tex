\section{Conclusion}
\label{sec:conc}

Bulk thermal conductivity predictions (QoI) for Si using NEMD are strongly dependent upon the
choice of nominal values of the SW parameters. However, the calibration process for the
parameters as discussed by Stillinger and Weber in~\cite{Stillinger:1985} was not
necessarily robust as it was based on a limited search focused on agreement with the 
available set of measured data and structural stability of the considered system. 
It did not account for the presence of measurement error, noise inherent in MD simulations,
and prior uncertainty associated with the parameters. It is therefore critical to
investigate the impact of uncertainty in the calibrated SW parameters on model predictions.
However, as highlighted in this work, it would be impractical to completely rely on
atomistic simulations for this purpose. Hence, in this study, we focused our efforts on
discovering the so-called active subspace that predominantly captures the variability
in bulk thermal conductivity predictions as a function of fewer independent variables
compared to the original input space. Efficient computation of the active subspace
was accomplished using an iterative gradient-free approach presented in 
Section~\ref{sec:method}. Based on the implementation of this approach and the results
presented in this work, the following conclusions can be drawn:
%
\begin{enumerate}
\item The active subspace computation using the iterative gradient-free approach
required only 25 NEMD simulations in the 7-dimensional input space using a tolerance, 
$\tau$~=~0.1.

\item A 1-dimensional active subspace was obtained indicating enormous scope for
computational gains using the framework presented in this work.

\item The L-2 norm of the error introduced by using a linear regression-fit as a surrogate to the
QoI variability in the active subspace was found to be approximately 0.1.

\item The PDF based on the 1-dimensional surrogate was observed to capture the mean, mode, and
the uncertainty in the QoI with reasonable accuracy. Hence, the local linear approximation
of the model output (see Algorithm~\ref{alg:lla}) seems like a promising technique for
such compute-intensive applications.

\item Global sensitivity analysis based on Sobol indices using the surrogate, and the
normalized activity scores yielded consistent results. Specifically, the SW parameters:
$\gamma$, $\alpha$, and $\lambda$ were found to be major contributors to the
uncertainty in the QoI.   

\end{enumerate}
%
The large uncertainty in the bulk thermal conductivity based on NEMD predictions due to
a 10$\%$ perturbation in the nominal SW parameter values in Figure~\ref{fig:level2} underscores
the importance of UQ in atomistic simulation predictions. Successful implementation of the
computational framework aimed at dimension reduction presented in this work is remarkably
encouraging. Extension of this work would involve development of novel, more efficient and
robust computational strategies facilitated by implementation to complex, higher dimensional
applications.
